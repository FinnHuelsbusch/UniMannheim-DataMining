\section{Application Area and Goals} \label{sec:applicationAreaGoals}

Heart disease is currently still one of the highest causes of mortality on earth \citep{nahar2013, kavitha2016, statistischesbundesamt2020}.
Given the successful application of data mining in other sectors e.g. banking and finance or marketing \citep{keles2017} possible applications in the medical industry are plentiful. Yet the healthcare sector is "information rich but knowledge poor" \citep{soni2011}. According to \citet{soni2011} medical datasets provide great potential for data mining to be used in clinical diagnosis.


The aim of this project was the application of data mining methods, more specifically classification methods, to predict whether or not a patient could suffer from a heart disease. The successful application could help doctors and medical staff with diagnosing patients by automatically analysing historical test result\todo{Denke, dass der konkrete Fall eher der ist, dass wir günstigere/simplere Mehtoden nehmen und dann eine Vorhersage treffen, wenn die positiv ist, dann machen wir komplexere analysen(mrt mit kontrastmittel)} data of the patient and give a prediction when a higher potential of heart problems arise. By doing this analysis patients flagged for potential heart disease could possibly be prioritised. Due to the immense amount of stress and long working hours medical personal are facing, having a standardized scheme looking at the data could be beneficial. 
In the past such approaches have already been tested and proven to be a good diagnostic option \citep{usharani2011}. \citet{jabbar2013} state that data mining techniques answer several important and critical questions related to healthcare and that they can improve the provision of quality services to patients.

This project report is based on the \say{Heart Disease Dataset} \citep{janosi1988} which, despite its age is still relevant given the fact that it consists of results of medical tests. In addition to that the validity is assumed because it is frequently used in contemporary research (see \cite{usharani2011, aha1988, nahar2013}).
