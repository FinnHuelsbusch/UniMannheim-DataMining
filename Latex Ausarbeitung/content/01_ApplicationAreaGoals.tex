\chapter{Application Area and Goals}

Heart disease is currently still one of the highest causes of mortality on earth \citep{nahar2013, kavitha2016, statistischesbundesamt2020}.
Given the successful application of data mining in other sectors e.g. banking and finance or marketing \citep{keles2017} possible applications in the medical industry are plentiful. Yet the healthcare sector is information rich but knowledge poor \citep{soni2011}. According to \citet{soni2011} medical data sets provide great potential for data mining to be used in clinical diagnosis.


This aim of this project was the application of data mining methods, more specifically classification methods, to predict whether or not a patient could suffer from a heart disease. The successful application could help doctors and medical staff with diagnosing patients by automatically analysing historical test result data of the patient and give a prediction when a higher potential of heart problems arise. By doing this analysis patients flagged for potential heart disease could possibly be prioritised. Due to the immense amount of stress and long working hours medical personal are facing, having an additional instance looking at the data could be beneficial. 
In the past such approaches have already been tested and proven to be a good diagnostic option \citep{usharani2011}. \citet{jabbar2013} state that data mining techniques answer several important and critical questions related to healthcare and that they can improve the provision of quality services to patients.

This project report is based on the \say{Heart Disease Data Set} \citep{janosi1988} which, despite its age is still relevant given the fact that it consists of results of medical tests. In addition to that the validity is assumed because it is frequently used in contemporary research (see \cite{usharani2011, aha1988, nahar2013}).





\newpage

\begin{itemize}
	\item Heart disease is one of the highest causes of mortality \citep{nahar2013, kavitha2016, statistischesbundesamt2020}
	\item DM is successfully applied in other sectors e.g. banking and finance or marketing \citep{keles2017} but healthcare is still "information rich" but knowledge poor". \citep{soni2011}
	\item explain how classification can be used in the medical field (explanation classification \citep{usharani2011})
	\item Aiding doctors with diagnosing patients by giving a classification
	\begin{itemize}
		\item “Data mining techniques have been widely used in diagnostic and health care applications because of their predictive power. Data mining algorithms can learn from past examples in clinical data and model the oftentimes non-linear relationships between the independent and dependent variables. The resulting model represents formalized knowledge, which can often provide a good diagnostic opinion.” ([Usha Rani, 2011, p. 2]) \citep{usharani2011}
		\item “Classification is a pervasive problem that encompasses many diverse applications. To improve medical decision making data mining techniques have been applied to variety of medical domains. Many health care organizations are facing a major challenge is the provision of quality services like diagnosing patients correctly and administering treatment at reasonable costs. Data mining techniques answer several important and critical questions related to health care.” ([jabbar et al., 2013, p. 86]) \citep{jabbar2013}
	\end{itemize}
	\item Data Mining helps to extract patterns in the process of knowledge discovery. DM provides new techniques which help the humans to analyze and understand large amounts of data for difficult and unsolved problems. \citep{usharani2011}
	\item Validity of our dataset: it is often used in the scientific field. \citep{usharani2011, aha1988, nahar2013}
	\item A faster and more precise detection of a possible heart disease will enable a more immediate treatment and thus may save more lives.
\end{itemize}

\section{Goals}

\begin{itemize}
	\item Giving doctors and medical staff a prediction of the medical status of the patient to increase awareness. medical history data contains huge amounts of test results and can be out of the scope of the examination. A automatic classification of a patients test values could increase the doctors attention to a more holistic overview. 
	\item Medical history data consists of a large number of tests required to diagnose a particular disease \citep{gupta2007}
	\item Using a supervised learning classification algorithm to learn from historical, labled data to derive a model that can classify new data.
\end{itemize}
