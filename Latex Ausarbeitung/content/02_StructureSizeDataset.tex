%\newpage
\section{Structure and Size of the Dataset} \label{sec:dataUnderstanding}

% minimum 1 page

The data used in the remainder of this paper comes from four individual datasets collected at different universities (Zurich, Budapest, Long Beach, Cleveland). The distribution of the target variable can be seen in \cref{table:datasets}
% PLACEHOLDER TABLE
\begin{table}[h]
    \label{table:datasets}
    \begin{footnotesize}
        \begin{tabular}{|l|l|l|l|}
            \hline
            \textbf{Publisher of data set}              & \textbf{\# of instances} & \textbf{Distribution target variable} \\ \hline
            Hungarian Institute of Cardiology, Budapest & 294                      & 106 / 188                             \\ \hline
            University Hospital, Zurich                 & 123                      & 115 / 8                               \\ \hline
            V.A. Medical Center, Long Beach             & 200                      & 149 / 51                              \\ \hline
            Cleveland Clinic Foundation, Cleveland      & 282                      & 125 / 157                             \\ \hline
        \end{tabular}
    \end{footnotesize}
\end{table}
\begin{center}
    \centering
    Distribution = heart disease / no heart disease
\end{center}
For the purposes of this paper, the four datasets are considered as one coherent dataset consisting of 77 attributes (33 numeric 42 categorical 1 constant) and a total of 899 measurements. 
Since not every source has collected all attributes, $31\%$ of all fields are filled with $-9$, which is equivalent to "not collected". \todo{Übergang} 
The attributes can be divided into the following categories. 
\begin{multicols}{2}
    \begin{itemize}
        \item Patient data
        \item Electrocardiogram
        \item Cardiac fluoroscopy
        \item Coronary angiograms
    \end{itemize}
\end{multicols}
The category of \textit{patient data} includes characteristics such as include \textit{age}, \textit{sex} or type of chest pain (\textit{cp}), whereas the category \textit{electrocardiogram} includes various electrocardiographic information during exercise like the peak blood pressure (\textit{tpeakbps}). \textit{Cardiac fluoroscopy} contains all measurements obtained from a \textit{Cardiac fluoroscopy} a medical measure that allows to see the flow of blood through vessels to evaluate the presence of blockages. An example for an attribute from this category is \textit{CA} which denotes the number of found vessels. The last category is \textit{Coronary angiograms}, which contains the results of the examination of the same name. The main attribute of interest is \textit{num} because it is the target variable that denotes whether a patient has a heart disease or not. An overview of all remaining attributes which were not named here is provided in our code documentation as well as the UCI Machine Learning Repository\citep{janosi1988}. In general, it is assumed that the collected attributes are of high quality because, apart from the individual characteristics described by the patient, as they are the result of standardized medical measurement procedures. Because of this, it is also assumed that a combination of the individual data sets is possible at all.   

In general, the data set is not sorted in any particular way, although for both the \textit{ECG} and the \textit{coronary angiograms}, the date on which the measurement was taken has been saved. 
But it is assumed that the date has no effect on the results, in addition, the time span covered, seven years, is too short to be able to deduce a general medical development. 








Our target variable is resembled by the attribute \textit{num} and encoded binary, with the value 1 indicating the diagnosis of a heart disease and the value 0 contradicting that indication. Looking at the distribution of the target variable we observe strongly varying distributions for the different datasets as specified in Table 1. The distribution in the combined dataset is relatively equal with 495 positive and 404 negative measurements, meaning a disease prevalence of roughly 55,1\%. When examining distribution of the other attributes we also notice uneven distributions. For example, the attribute \textit{sex}´s distribution contains 78\% male and 22\% female patients in our dataset, contributing to the Gender Data Gap prevalent in medicine. We also notice further uneven distributions across our datasets. Whilst highlighting them in the report would exceed the frame, we address these deviations in the preprocessing of our dataset and documented them.

Although there is time-based data in the dataset, it is limited to only one entry per patient. Therefore, the dataset is not suitable for a time series analysis.

Lastly, verifying data quality also acts as an important part of Data Understanding. When checking for missing data we observed multiple interesting results. We observe a high number of cells with missing values, lacking 21397 or 30,8\% of values. Some individual attributes in particular have many missing values like history of diabetes (\textit{dm}) with roughly 90\% missing values. These missing values are sometimes dataset-specific (e.g. \textit{painloc} is mainly missing in the Cleveland dataset) but also cross-dataset (e.g. \textit{dm}).

Furthermore, when checking whether meanings of attributes and their contained values fit together, we observe some irregularities. For example, looking at the attribute cholesterin (\textit{chol}), the values listed seem to be unrealistic (e.g. many instances with 0 serum cholesterol in mg/dl).

After obtaining an understanding of properties and meaning of our data, we perform initial analysis to explore additional inisights of our dataset

