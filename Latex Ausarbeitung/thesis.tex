\documentclass[11pt,titlepage,oneside,openany]{book}
\usepackage{times}

\usepackage[
	backend = biber,                		% Verweis auf biber
	language = auto,						% Sprache wird automatisch festgelegt
	style = authoryear,                	% Nummerierung der Quellen mit Zahlen
	bibencoding = utf8,						% UTF8 wird in biblatex aktiviert
	sorting = none,                 		% none = Sortierung nach der Erscheinung im Dokument
	sortcites = true,               		% Sortiert die Quellen innerhalb eines cite-Befehls
	block = space,                  		% Extra Leerzeichen zwischen Blocks
	hyperref = true,                		% Links sind klickbar auch in der Quelle
	doi=true,                      			% DOI anzeigen
	isbn=true,                     			% ISBN anzeigen
	alldates=short                  		% Datum immer als DD.MM.YYYY anzeigen
]{biblatex}
\addbibresource{thesis-ref.bib}	

\usepackage[textsize=tiny]{todonotes}
\usepackage{graphicx}
\usepackage{latexsym}
\usepackage{amsmath}
\usepackage{amssymb}

%\usepackage{ntheorem}

% \usepackage{paralist}
\usepackage{tabularx}

% this packaes are useful for nice algorithms
%\usepackage{algorithms}
\usepackage{algorithmicx}

% CUSTOM PACKAGES
% this package will help with German ä ü ö etc.
\usepackage[utf8]{inputenc}


% -- Varioref
\usepackage{varioref}


% -- Hyperref
\usepackage{hyperref}
\hypersetup{%
	linktocpage=true, 				% Nicht der Text sondern die Seitenzahlen in Verzeichnissen klickbar
	bookmarksnumbered=true 			% Überschriftsnummerierung im PDF Inhalt anzeigen.
}


% -- Cleverev
\usepackage[noabbrev]{cleveref}   	% Kein Erkennen von Abkürzungen 


% well, when your work is concerned with definitions, proposition and so on, we suggest this
% feel free to add Corrolary, Theorem or whatever you need
\newtheorem{definition}{Definition}
\newtheorem{proposition}{Proposition}


% its always useful to have some shortcuts (some are specific for algorithms
% if you do not like your formating you can change it here (instead of scanning through the whole text)
\renewcommand{\algorithmiccomment}[1]{\ensuremath{\rhd} \textit{#1}}
\def\MYCALL#1#2{{\small\textsc{#1}}(\textup{#2})}
\def\MYSET#1{\scshape{#1}}
\def\MYAND{\textbf{ and }}
\def\MYOR{\textbf{ or }}
\def\MYNOT{\textbf{ not }}
\def\MYTHROW{\textbf{ throw }}
\def\MYBREAK{\textbf{break }}
\def\MYEXCEPT#1{\scshape{#1}}
\def\MYTO{\textbf{ to }}
\def\MYNIL{\textsc{Nil}}
\def\MYUNKNOWN{ unknown }
% simple stuff (not all of this is used in this examples thesis
\def\INT{{\mathcal I}} % interpretation
\def\ONT{{\mathcal O}} % ontology
\def\SEM{{\mathcal S}} % alignment semantic
\def\ALI{{\mathcal A}} % alignment
\def\USE{{\mathcal U}} % set of unsatisfiable entities
\def\CON{{\mathcal C}} % conflict set
\def\DIA{\Delta} % diagnosis
% mups and mips
\def\MUP{{\mathcal M}} % ontology
\def\MIP{{\mathcal M}} % ontology
% distributed and local entities
\newcommand{\cc}[2]{\mathit{#1}\hspace{-1pt} \# \hspace{-1pt} \mathit{#2}}
\newcommand{\cx}[1]{\mathit{#1}}
% complex stuff
\def\MER#1#2#3#4{#1 \cup_{#3}^{#2} #4} % merged ontology
\def\MUPALL#1#2#3#4#5{\textit{MUPS}_{#1}\left(#2, #3, #4, #5\right)} % the set of all mups for some concept
\def\MIPALL#1#2{\textit{MIPS}_{#1}\left(#2\right)} % the set of all mips





\begin{document}

\pagenumbering{roman}
% lets go for the title page, something like this should be okay
\begin{titlepage}
	\vspace*{2cm}
  \begin{center}
   {\Large Analysis of Heart Disease Data\\}
   \vspace{2cm} 
   {Project Report\\}
   \vspace{2cm}
   {presented by\\
   	Club der toten Dichten (Team 12)\\
    Marie-Christin Häge, 1913888 \\
    Finn Hülsbuch, 1913864 \\
    Thilo Dieing, 1692328 \\
    Lasse Lemke, 1914420 \\
    Eric Jacquomé, 1903834 \\
    Timotheus Gumpp, 1913876 \\
   }
   \vspace{1cm} 
   {submitted to the\\
    Data and Web Science Group\\
    Prof.\ Dr.\ Heiko Paulheim\\
    University of Mannheim\\} \vspace{2cm}
   {December 2022}
  \end{center}
\end{titlepage} 

% no lets make some add some table of contents
%\tableofcontents
%\newpage

%\listofalgorithms

%\listoffigures

%\listoftables

% evntuelly you might add something like this
% \listtheorems{definition}
% \listtheorems{proposition}

\newpage


% okay, start new numbering ... here is where it really starts
\pagenumbering{arabic}

% =============================================
% 			HERE GOES THE CONTENT!
% =============================================

\chapter{Application Area and Goals}

\section{Application Area}

adfasdfasdfads

Test Zitat \cite{statistischesbundesamt2022}


\section{Goals}

asfasfasfs
%\newpage
\section{Structure and Size of the Dataset} \label{sec:dataUnderstanding}

% minimum 1 page

The data used in the remainder of this paper \todo{"in this paper" -> vorher haben wir ja auch schon über die Daten gesprochen}comes from four individual datasets collected at different universities (Zurich, Budapest, Long Beach, Cleveland). The distribution of the target variable can be seen in \cref{table:datasets}.
% PLACEHOLDER TABLE
\begin{table}[h]

    \begin{footnotesize}
        \begin{tabular}{|l|l|l|l|}
            \hline
            \textbf{Origin of data set}              & \textbf{\# of instances} & \textbf{Distribution target variable} \\ \hline
            Hungarian Institute of Cardiology, Budapest & 294                      & 106 / 188                             \\ \hline
            University Hospital, Zurich                 & 123                      & 115 / 8                               \\ \hline
            V.A. Medical Center, Long Beach             & 200                      & 149 / 51                              \\ \hline
            Cleveland Clinic Foundation, Cleveland      & 282                      & 125 / 157                             \\ \hline
        \end{tabular}
    \end{footnotesize}
    \begin{center}
        \centering
        Distribution = heart disease / no heart disease
    \end{center}
    \caption{Content of the dataset}
    \label{table:datasets}
\end{table}



For the purposes of this paper, the four datasets are considered as one coherent dataset consisting of 77 attributes (33 numeric, 42 categorical, 1 constant) and a total of 899 measurements.
The attributes can be divided into the following categories:
\begin{multicols}{2}
    \begin{itemize}
        \item Patient data
        \item Electrocardiogram
        \item Cardiac fluoroscopy
        \item Coronary angiograms
    \end{itemize}
\end{multicols}
The category of patient data includes characteristics such as \textit{age}, \textit{sex} or type of chest pain (\textit{cp}), whereas the category electrocardiogram includes various electrocardiographic information obtained during an exercise electrocardiogram like the peak blood pressure (\textit{tpeakbps}). Cardiac fluoroscopy contains all measurements obtained from a cardiac fluoroscopy, a medical measure that allows to see the flow of blood through vessels to evaluate the presence of blockages. An example for an attribute from this category is \textit{ca} which denotes the number of found major vessels. The last category is coronary angiograms, which contains the results of the examination of the same name. The main attribute of interest of this category is \textit{num} because it is the target variable that denotes whether a patient has a heart disease or not. In the raw data, patients without a heart disease are shown with a value of 0 for \textit{num}, while patients with heart disease are shown with a value  $\geq 1$.
An overview of all remaining attributes which were not named here is provided in our code documentation as well as the UCI Machine Learning Repository \citep{janosi1988}. In general, it is assumed that the collected values are of high quality, as they are the result of standardised medical measurement procedures, apart from the individual characteristics described by the patient, such as the location or type of pain. For this reason, it is assumed that a combination of the individual data sets is possible. This is also true because the measurements are not sorted in a certain way, neither in the individual data sets nor in the combined dataset.

The dataset contains dates for the conducting of the electrocardiogram and coronary angiograms. These dates are unevenly distributed over a period of 7 years. We consider these dates irrelevant because we assume that the date of an examination does not affect its outcome as the period is to short to show evolutionary or general health changes in the society. Therefore a time series analysis is not possible with this dataset.

Finally, the distribution of the different variables in the combined dataset was analysed. The target variable was found to be reasonably symmetrically distributed with 495 positive and 404 negative measurements, implying a disease prevalence of about 55.1\% of the examined patients. Non-representative distributions were also found, e.g. \textit{sex} is non-representatively distributed with 78\% males and 22\% females. The same applies to \textit{age}, which is distributed similarly to a normal distribution in the range 28 to 77.

While assuming a high quality of the existing values, we found that $30,8\%$ of all fields are lacking measurements. This is attributed to the fact that not all universities have carried out all measurements. For example the Cleveland subset lacks the location of chest pain. Some individual attributes in particular have many missing values like history of diabetes (\textit{dm}) with roughly 90\% missing values. Whereby only positive cases may have been entered here, since 95\% of all filled cells show a diabetes disease which again is not representative. Furthermore, when checking whether meanings of attributes and their contained values fit together, we observe some irregularities. For example, the values 0 for cholesterol (\textit{chol}) and blood pressure \textit{testbps} are not compatible with life. Therefore, it is assumed that these values are misreported non-surveys, which are dealt with in the following chapter Preprocessing.

\newpage
\section{Preprocessing}
We approach preprocessing in two steps. In the first step we clean the data set based on knowledge we obtained in the Data Understanding section and further analysis. 
Secondly, we try many different approaches to transform the data, where we can not be certain about the best method. 

\subsection{Cleaning}
Firstly, we implement a custom loading function to transform the 4 datasets into a csv format, so we can use it further on.

We remove the false predictors lmt, ladprox, laddist, diag, cxmain, ramus, om1, om2, rcaprox and rcadist, as our target variable num is a combination of these according to the UCI. %Nochmal angucken mit Finn.

Furthermore, the features restckm, exerckm, thalsev, thalpul, lvx1, lvx2, lvx3, lvx4, lvf, dummy and junk are considered irrelevant by the UCI, so we also drop them. Other irrelevant attributes we remove are IDs(id,ccf), constants(name, earlobe) and dates of medical examinations (ekgmo, ekgday, ekgyr, cmo, cday, cyr). We consider these dates irrelevant because we assume that the date of  an examination does not affect its outcome. 

We drop the feature pncaden because it is the sum of painlox, painexer and relrest and therefore contains no additional information. 

The features cp, restecg, slope, ca and restwm were oneHotEncoded as they represent categorical values.

When checking for inconsistencies between features, we detected that thaltime is sometimes lower than thaldur. As thaltime describes the moment ST depression is noted within the exercise, it has to be lower than the duration of the exercise thaldur. We replace thaltime by \texttt{NaN} in all 17 instances that do not satisfy this criterion. 

Also, the maximum heart rate (thalach) was replaced with \texttt{NaN} if it was lower than the heart rate at rest (thalrest).

\begin{figure}[h]
	\centering
	\includegraphics[width=0.7\textwidth]{images/dataDistribution.png}
	\caption{data distribution of all numeric elements}
	\label{fig:dataDistribution}
\end{figure}
As shown in \ref{fig:dataDistribution} we created a normalized box plot of all numeric features to check for outliers. The features trestbps and trestbpd show extreme outliers with a value of 0. These are incorrectly specified \texttt{NaN}s and are therefore replaced by \texttt{NaN}. All other outliers are not as extreme and come in groups. As the data contains sick persons, values diverging from the norm are expected. For these reasons we decided to keep these outliers as they can be a strong indication of a heart disease.

The remaining features were analysed regarding their pearson correlation. Only two pairs of features with substantive amount of data (less than 75\% \texttt{NaN}s) have a very strong correlation (\>80\%).  

These are cp\_4$\leftrightarrow$painexer and rldv5$\leftrightarrow$rldv5e. The highest correlation is between cp\_4 and painexer. The feature cp\_4, which was oneHotEncoded from the categorical variable cp, describes whether the patient has no chest pain at rest. Painexer describes whether the patient only has pain when exercising. 

Concluding from the high correlation between the EKG amplitude when resting (rldv5) and the EKG amplitude when exercising (rldv5e), we decided to create a new feature(rldv5\_diff) by using the difference between these. We did the same with resting heart rate and maximum heart rate (thal\_diff = thalach - thalrest). 

Furthermore, we enrich the feature smoke using years and cigs. Hereby, we reduce the number of \texttt{NaN}s from 74\% to 43\%. 

\subsection{Hyperparameter Tuning }
For hyperparameters and methods where we could not be certain, we try out multiple different combinations.

Firstly, we try binning the feature age. We choose either 2 or 5 bins or no binning at all. We decided to use equal width binning so that the age groups are simpler and more intuitive to a doctor.

\begin{figure}[h]
	\centering
	\includegraphics[width=0.7\textwidth]{images/percentageToBeDropped.png}
	\caption{Number of features and values to be imputed by number of \texttt{NaN}s}
	\label{fig:percentageToBeDropped}
\end{figure}
Figure \ref{fig:percentageToBeDropped} shows the number of features, that have less than X\% missing data and how many cells we would need to impute if we included them. It becomes apparent that there are certain steps where the number of features goes up a lot. To decide when a feature is included based on the number of missing values, we try the steps 0, 4, 8, 20, 35, 60, 75 and 100 \% in the model. They are shown as vertical lines in the graph. Additionally, we decided to drop features based on their correlation. For this we decided to use the steps [X,X,X,X].

To impute the missing data we use a simple imputer. Missing values are replaced by the mean, median or mode of the feature. We decided against using a KNN imputer, because it is computationally much more expensive. The iterative imputer is not used, as it is still experimental and therefore subject to change.

To account for the different ranges of the features different scalers are tried out. We only use scalers that are applicable for all integers as some features contain negative values. 
We compare the MaxAbsScaler, MinMaxScaler, PowerTransformer, RobustScaler, Standardscaler and Normalizer. As the hyperparameters of most scalers turn on or off core functionalities of the scaler, we decided to only tune the hyperparameter norm of the Normalizer with the norms l1, l2 and max.

To account for the different amounts of healthy and unhealthy patients we try oversampling and undersampling in the training data in comparison to passing the values through.


%The \textbf{MaxAbsScaler} scales the values of each feature by the maximum absolute value. Therefore, all values in [-1,1] can occur after scaling. \newline
%Using the \textbf{MinMaxScaler} results in values in [0,1] by shifting by the negative minimum and scaling by $maximum - minimum$\newline
%The values were normalized using the norms l1,l2 and max in the \textbf{Normalizer}. \newline
%The \textbf{PowerTransformer} alters the data to represent a gaußian like distribution. This is mostly used if heteroskedasticity occurs in the data. It was tried to fit the data to a standard normal distribution and without shifting and scaling by its mean and variance. \newline
%The \textbf{RobustScaler} was tried with and without scaling by the interquartile range and with and without shifting beforehand.\newline
%The \textbf{Standardscaler} was used with and without shifting by the mean and scaling by the standard deviation.\newline
%It was also tried, manly for comparison, to \textbf{passthrough} the values as they are.\newline
\section{Datamining} \label{sec:datamining}



%\newpage
\section{Results} \label{sec:results}

In order to be able to critically assess the result with the decision tree as the best model, a comparison is made with existing evaluations. Therefore, a search for papers, articles and competitions working with the dataset which describe their approach well is conducted. In doing so,  it became clear that the used approach is not widespread. The dataset is frequently used, but not as a whole. Often only one sub-dataset, mostly Cleveland, is used.  In comparison to the works that work exclusively on the Cleveland dataset, our best model is surpassed in every respect (\cite{Ayatollahi2019,alotaibi2019, uyar2017}). It should be noted that other models might not be generalisable as the models in this paper, as they may represent noise from the respective data set. Furthermore, it should be noted that the Cleveland dataset is a very pure dataset compared to the other datasets and hardly contains any missing values. 
A comparison of the most important features is also not possible as the only work that that used every feature also included the false predictors in their models \cite{garate-escamila2020}.

When compareing the best model against the majority class baseline with a recall of 1, the best model is surpassed. Though this is manly due to the poor selection of recall as key metric as it does not reflect the usefulness of the model because it ignores the performance on negative values. If F! is used as a metric the Decision tree is able to outperform the baseline. 

To conclude whether the project helps doctors on diagnosing possible heart diseases more easily, certain limitations into needed to be taken into account. Type 2 errors in disease prediction are particularly problematic because a sick patient is mistakenly found to be healthy and therefore might not receive the correct treatment. Contrary the Type 1 error, might lead to healthy patients getting medication they do not need.   When considering the actual application of the model in practice, it is important to overcome the "black box" of machine learning for users. For this, explainable AI models like the Decision Tree help to be interpretable and trustworthy even for laymen (see \cref{fig:DecisionTree}). Therefore, the Decision Tree is a well chosen model. Though the decisions made by the Tree (old man with asymptomatic chest pain are likely to have a heart disease) is well known for years. Therefore, applying the model in the real world would not make sense as there is no added value for the user. 

% =============================================
% 			END OF CONTENT!
% =============================================

\vspace{2cm}
\begin{small}
  \printbibliography
\end{small}

\pagestyle{plain}

\end{document}
